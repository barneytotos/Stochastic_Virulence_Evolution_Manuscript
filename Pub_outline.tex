
\section*{Background}

\subsection*{Paragraph 1: Introduce the problem of parasite evolution}

Competition between host and parasite exerts strong selection pressure on both parasite exploitation strategy and host defense strategy \citep{SchneiderandAyres2008, AyresandSchneider2012, KutzerandArmitage2016}. The resulting evolution often occurs at ecological time scales \citep{DayandProulx2004, Bolkeretal.2010, Lion2018}, and sometimes in a predictable way \citep{Ebert1998, Pugliese2002, Laineetal.2008, Frickeletal.2016}. Often, rapid evolution occurs after a parasite jumps species or emerges in a new host populations; examples of rapid parasite evolution following invasion include: WNV in North American birds \cite{Beasleyetal.2003}, MYXV in Eurpoean rabbits \citep{FennerandMarshall1957}, \emph{Plasmodium falciparum} in House finches \cite{Flemingetal.2018}, HIV in Humans \cite{Fraseretal.2007}, $\lambda$ phage in \emph{E. coli.} \cite{Berngruberetal.2015}, \emph{phycodnaviridae} family dsDNA viruses in algae (\citep{Frickeletal.2016, Frickeletal.2018}, and \emph{Pasteuria ramosa} in \emph{Daphnia sp.} \citep{Duffyetal.2009}. 

\subsection*{Paragraph 2: Deterministic vs Stochastic representations of the problem}

In the past four decades, models for parasite evolution informed by these examples have revealed generalities in parasite evolution and helped to explain parasite exploitation strategy in specific systems \citep[for a review see][]{Cressleretal.2016}. However, most mathematical models of parasite evolution assume both deterministic evolution in an infinite population, though parasite exploitation effects host population size \citep{Frickeletal.2016, Papkouetal.2016, Frickeletal.2018}. If stochasticity is used it all, often stochasticity's only role is to provide raw material for adaptation via mutation. Additionally, most of these models are restricted to long term evolution, employing a separation of separation of ecological and evolutionary time scales, despite common rapid evolution (but see \citealt{Bolkeretal.2010}, \citealt{Lion2018}, \citealt{Parsonsetal.2018}). 

These simplifications are often made in order to obtain analytic solutions, which are usually more easily generalizable; solutions from stochastic simulations are likely to be tied more closely to specific systems and can potentially lead to less powerful conclusions. However, because many researchers have recently been calling for more of an emphasis on a case-study approach \citep{BullandLauring2014, AlizonandMichalakis2015, Cressleretal.2016}, a reduced ability to make widely generalizable conclusions may be a minimal drawback. Work on evolution in finite populations is also restricted by a lack of mathematical theory for deriving analytic solutions for evolution in finite populations on ecological time scales \citep{Lion2018}, though \citet{Parsonsetal.2018} have recently derived analytic solutions for parasite evolution in finite populations using \emph{Stochastic Adaptive Dynamics}. 

\subsection*{Paragraph 3: Introduce the problem of a priori assuming a simple tradeoff}

Models often also collapse high-dimensional host-parasite interactions to two axes: parasite virulence (the rate of host mortality) and parasite transmission rate, which jointly determine parasite fitness \citep{AndersonandMay1982, Ewald1983, AlizonandMichalakis2015, Cressleretal.2016}. This \emph{a priori} assumption of a tradeoff results in a model where parasites move only \emph{on} their optimization frontier, which is unrealistic for any organism in a new environment. In this paradigm it is simultaneously 
impossible to understand how a parasite evolves \emph{to} this frontier because parasite traits apart from virulence and transmission are taken as a ``black box'' \citep{Alizonetal.2009, BullandLauring2014, AlizonandMichalakis2015}.

\subsection*{Paragraph 4; Our objective}

In an attempt to consider parasite virulence evolution from a broader perspective, and in particular to understand the diversity of virulence found in natural communities, we explore two stochastic, discrete-population models for parasite evolution: (1) Parasite evolution in transmission and virulence (modeled here as parasite-induced mortality rate) in the absence of a virulence-transmission tradeoff; (2) Parasite evolution when parasite transmission is constrained by virulence, but where parasites can evolve compensatory mechanisms to stay near the tradeoff frontier. 

\subsection*{Paragraph 5: Broad strategy prior to the methods section}

We analyze our model for the mechanics of parasite exploitation using three different modeling approaches: a biologically realistic scenario of parasite evolution in finite populations using a discrete time stochastic simulation-based model, as well as a deterministic reaction-diffusion (differential equation based) model (RD) that captures both selection (advection) and mutation (diffusion) and an adaptive dynamics representation of the problem (AD). These deterministic approaches serve as two forms of ``null-models'' against which we compare the results of our stochastic model. Using these three models we aim to understand the effects of small populations and stochasticity on transient evolution, time to the adaptive peak, and periodic departures of the population from its adaptive peak. We examine when and what interesting biological patterns emerge as a function of increasing model complexity; that is, what we can learn about parasite evolution using a stochastic model that is missed when using a simpler framework.
