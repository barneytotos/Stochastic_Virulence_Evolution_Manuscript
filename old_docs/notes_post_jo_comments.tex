These statements could maybe still be used somehwere, possibly in the abstract if reduced

-- Parasite exploitation of their host is a complex antagonistic interaction: parasite genes promote replication, evasion of host immunity, competition with other organisms within the host, and ultimately transmission to new hosts \cite{Bushetal.2001}; hosts strive to remove parasites and/or reduce the negative fitness impacts of parasite exploitation using a diversity of immune mechanisms \citep{Soaresetal.2017}.

-- However, as many have pointed out, tradeoff theory is a simple model \citep{BullandLauring2014, AlizonandMichalakis2015, Cressleretal.2016}; it is not only unlikely that a parasite will evolve only along the tradeoff curve, it is unlikely that it will always find itself on the tradeoff curve, or for that matter, even be able to make it to its tradeoff curve. The tradeoff curve is by definition an evolutionary frontier; it is a certainty, especially for a parasite that has experienced a host jump, that a parasite invades initially far from this frontier.


These statements have more or less been rewritten and are likely not needed anymore.

-- Despite this simplification, tradeoff theory has revealed a number of important generalities about parasite evolution (for a review see \citealt{Cressleretal.2016}), and has helped to explain intermediate virulence seen in a moderate number of host-parasite systems including $\lambda$ phage in \emph{E. coli} \citep{Berngruberetal.2015}, HIV in humans \citep{Fraseretal.2007}, \emph{Ophryocystis} in monarch butterflies \citep{DeRoodeetal.2008}, Cauliflower Mosaic Virus in \emph{Brassica rapa} \citep{Doumayrouetal.2013}, and Myxoma virus in European rabbits \citep{AndersonandMay1982}.

-- Finally, stochastic models can be difficult to parameterize, and qualitative dynamics (e.g. bifurcations) are harder to pin down accurately.

-- Extensions to tradeoff theory that introduce additional axes of complexity such as within-host parasite dynamics, host evolution, or spatial structure (for a review see \citealt{Cressleretal.2016}), can improve qualitative predictions for a wider range of host-parasite systems. 


These statements are not needed in the introduction, but are needed in the paper. Can likely be moved to the methods:

-- It is assumed that parasite transmission is a decreasing function of parasite exploitation; when parasite transmission is a convex function of parasite virulence, parasites maximize fitness at an intermediate level of virulence \citep{AndersonandMay1982, Alizonetal.2009}. 

-- For example, parasite virulence is regularly described (but often not modeled) as some increasing function of the semi-mechanistic trait ``replication rate'' \citep{ AlizonandvanBaalen2005, Bolkeretal.2010}, though parasite virulence is often directly taken as a trait \citep{AndersonandMay1982, AlizonandMichalakis2015}. 

-- We collectively model compensatory mutations using a quantitative trait we call parasite parasite \emph{tuning}, which evolves to allow a parasite to maximize its transmission potential defined by the tradeoff curve frontier.


