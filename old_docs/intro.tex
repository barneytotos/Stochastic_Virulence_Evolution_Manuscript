\section*{Introduction}

Competition between a host and parasite often generates strong selection pressure on parasite exploitation strategy and host defense strategy. Parasites are assumed to be able to adjust their aggressiveness, while hosts can invest in resistance (reducing parasite load) or tolerance (decreasing fitness costs for a given pathogen load) (Boots 2008, Adelman and Hawley 2017). A parasite's investment in aggressiveness and host's investment in resistance and tolerance jointly determine host mortality rate (parasite virulence), host recovery rate, and parasite transmission rate (Carval and Ferriere 2010). Predicting these quantities is important for predicting host-parasite dynamics (Restif and Koella 2003, Boots 2008). For example, the evolution of host resistance is predicted to generate Red Queen cycles (fluctuations in both host and parasite density and dominant phenotype), while tolerance evolution is expected to lead to selective sweeps but to not affect parasite prevalence (Restif and Koella 2003), which can lead to high mortality in naive hosts when they become infected (Miller et al. 2006). Joint parasite and host strategies are also important for predicting the impacts of intervention strategies such as vaccination (Gandon et al. 2001) \jw{I don't feel like you need another example} and ... (culling?).

Much work sets out to explain the structure of host and parasite communities; however, this work can capture only a narrow class of host-parasite interactions... 

\subsection*{Adaptive Dynamics}

The majority of our understanding of host and parasite evolution comes from models using the ``Adaptive Dynamics'' (AD) framework, which is built on two primary assumptions: separation of time scales, and a cleanly defined optimization criterion, most often the maximization of $\rzero$. The separation of time scales means that the epidemiological time frame is much faster than the evolutionary time frame... This necessitates that a single mutation arises at a time, and resolves itself through competitive exclusion (based on the optimization criteria)... This framework can be boiled down to (essentially) write down a tradeoff (see below) and find the ESS. 

\subsection*{Eco-evolutionary dynamics}

A second class of models assumes that evolution and ecology (epidemiology) operate on similar time scales, and feed back upon each other to determine the evolutionary trajectories of host and parasite (Day and Proulx 2004). In eco-evolutionary (eco-evo) models of host-parasite interactions, changes in host density and host infection prevalence change the evolutionary pressure on both host and parasite... Eco-evo models are better suited for transient virulence evolution, such as at the beginning of epidemics (Bolker et al. 2010). Lion (2018) intuitively describes the difference between AD and eco-evo models. Put simply, the environment (the ecology), call this E, which includes everything that is not the evolving entity. If E is at an equilibrium and does not affect the evolution of the focal population, we are in AD territory. Otherwise, if E is assumed to change, which changes how evolution is operating, we are in the realm of eco-evo models. 

\subsection*{Virulence evolution}

Regardless of the mathematical framework, most models of parasite evolution assume \emph{a priori} that parasites are evolving along a tradeoff between transmission rate and virulence (or host recovery) (but see Alizon and vanBaalen 2005 and King et al. 2009 for the emergence of a tradeoff from within-host dynamics), which gives rise to highest fitness (most often $\rzero$) at intermediate transmission rate. While empiricists have documented the presence of trade-offs in other natural host-parasite systems (e.g $\lambda$ phage in \textit{E. coli}: \citealt{Berngruberetal.2015}; HIV in humans: \citealt{Fraseretal.2007}; \textit{Ophryocystis} in monarch butterflies: \citealt{DeRoodeetal.2008}; Cauliflower Mosaic Virus in \textit{Brassica rapa}: \citealt{Doumayrouetal.2013}, the reliance on tradeoff theory has recently been criticized (Alizon and Michalakis 2015, Cressler et al. 2016). 

\subsection*{Host defense evolution}

The evolution of host defense responses to parasite exploitation have been studied using mathematical models for decades (Anderson and May 1982, Roy and Kirchner 2000, Best et al. 2014). This vast modeling literature has historically considered host resistance (reducing pathogen load) as the primary mechanism of host defense; more recent theoretical and empirical studies (theoretical: Carval and Ferriere 2010, Best et al. 2014; empirical: Raberg et al. 2007, Athanasiadou et al. 2015, Rangan et al. 2016) have considered host tolerance (decreasing the impact of a given pathogen load on host fitness) as a host strategy for mitigating the effects of pathogen infection. (The plant-pathogen literature has focused on tolerance for nearly a century [Foroni et al. 2013], but the animal-pathogen literature has only recently turned in this direction.) 

Most models consider parasite evolution in response to a non-evolving host or host defense evolution in response to a non-evolving parasite. Other models consider coevolution of a single host defense (resistance or tolerance) and pathogen virulence (the loss of fitness, or rate of mortality, of infected hosts); however, at least one study has employed a full coevolutionary model, considering evolution of both resistance and tolerance of hosts as well as evolution of parasite virulence (Carval and Ferriere 2010). 

In addition to variation among studies in what traits are assumed to be evolving, models vary in their assumptions about the genetic architecture of host-parasite interactions; the time scale on which ecology (epidemiology) and evolution occur; whether or not the effects of finite population size are considered; and the mode of selection. For a good description of the full range of genetic assumptions (gene-for-gene: GFG; matching allele: MA; and quantitative genetic: QG approaches) see Best et al. (2014). For a review on options for modeling evolution see Lion 2018; and for a more focused review on optimization criteria see Lion and Metz 2018.

Nearly all combinations of this multitude of possibilities have been explored mathematically. However, we have only very recently begun to understand the complex genetic and physiological pathways of both host resistance and tolerance to exploitation (Glass 2012, Medzhitov et al. 2012, Soares et al. 2017, McCarville and Ayres 2018). For this reason many of the assumptions that these models make about the shapes of cost functions and benefits of host resistance and tolerance and the correlations between them have been based on logic and conjecture alone. Yet, many conclusions have proven to be robust to a wide range of assumptions: tolerance rarely leads to genetic polymorphism (Roy and Kirchner 2000, Boots et al. 2009, Carval and Ferriere 2010); accelerating cost functions promote single strategies, while linear and decelerating cost functions can allow mixed strategies (Foroni et al. 2004, Restif and Koella 2004, Carval and Ferriere 2010); resistance is likely to result in temporal fluctuations in host genotype (Red Queen dynamics) (Sasaki 2000; Agrawal and Lively 2002); tolerance is less likely (but can) result in host counter-adaptation ((Boots and Bowers 1999; Roy and Kirchner 2000; Miller et al. 2005, Best et al. 2008); Hosts are expected to invest in tolerance against extremely aggressive parasites (Restif and Koella 2004, Carval and Ferriere 2010). 

Other conclusions are more model-specific due to slightly more esoteric modeling decisions: fixed parasite virulence selects for resistance but plastic virulence selects for tolerance (Soler and Soler 2017); tolerance is favored until a critical threshold of parasite load (Moreno-Garcia et al. 2014); Can add more here...

Critically, we currently lack sufficient data to test most of these possibilities. However, the diversity of pathways that could contribute to tolerance discussed in Soares et al. (2017) combined with previous modeling results indicating that tolerance is a viable option suggests that the paucity of evidence for tolerance is likely driven in large part by our recent search for tolerance and our ability to detect tolerance. Additionally, Soares et al. (2017) indicates that whether tolerance or resistance is likely to evolve may be influenced by the traits of a parasite. Yet, the majority of evidence of host defense evolution in both natural and laboratory systems involve host resistance evolution: European rabbits to the myxoma virus (MYXV) (Kerr et al. 2015); House finch to mycoplasma (Bonneaud et al. 2011); others.... --- Blue crab evolution of \emph{tolerance} to \emph{Mytilicola intestinalis} (Feis et al. 2016). A number of laboratory experiments have found \emph{de novo} evolution of host resistance to parasite infection (Bohannan and Lenski 2000, Van der Linden et al. 2013, Frickel et al. 2016, Frickel et al. 2018).

\subsection*{What are we missing}

Despite this vast literature, a number of questions remain, namely, do the expectations for parasite and host evolution from AD and more narrow eco-evo (infitine population size) models arise in a more realistic scenario of finite population sizes and eco-evo feedbacks that do not restrict evolution of host or parasite to a single resident and single invading strain. Here we ask What structure arise from a single host and parasite ancestor as a transient state or at a stationary equilibrium due to simple evolutionary rules. From this simple case we can begin to build expectations for what will evolve based on known traits that lead to predictable departures from a baseline model.

Many questions regarding the evolution of host defense strategy remain entirely unresolved... Why is resistance seen so much more frequently than tolerance? A non-exhaustive list: First, we haven't been looking for tolerance for very long in the animal literature; Second, tolerance is harder to detect than resistance for two reasons: statistical power is lower to detect variation in slopes than variation in intercepts, and since a beneficial tolerance allele is expected to selectively sweep, we may fail to detect changes if we haven't been monitoring populations over appropriate time scales; Third, standing genetic variation for tolerance may be lower than for resistance, leading to slower evolutionary change \bb{why? because of pop gen/sweeps?}; Fourth, tolerance may either be more costly, or less beneficial than resistance \bb{this explanation is at a different level from the others; begs the question slightly (i.e, what would the mechanistic reason be for this?}.

 One reason for the lack of work on the evolution of finite populations on ecological time scales is a lack of mathematical theory. Lion (2018) writes: ``Analytical progress is possible only through additional methodological assumptions, generally taking the form of a separation of time scales''. However, progress can still be made with simple simulations, especially in the case of a near-neutral null model that can serve as a comparison. The question becomes whether additional modeling studies provide further insight into this problem. Despite the vast literature we argue that more insight can be gained by additional modeling studies for a number of reasons: First, to build a comprehensive understanding of why resistance or tolerance evolves in a given host-parasite system, a null model is needed for comparison to illuminate mechanisms for why a given pathway was favored. \mk{Something about returning to the building blocks}. Neutral theory in ecology has proven to be useful for a similar reason... \mk{I think...}. If resistance or tolerance evolution is expected to be as context dependent as is becoming more evident, a number of comparisons will be needed to start to build a comprehensive picture for the traits that favor one pathway over the other. Second; despite fast evolution during epidemics, most existing modeling studies do not provide insight into the evolution of host and parasite during the course of an epidemic. The majority of studies employ a separation of time scales between ecological (epidemiological) and evolutionary time scales that is a departure from how evolution during an epidemic operates (Lion 2018); the majority of studies assume parasites maximize fitness by maximizing $\rzero$ (which will feed back to impact how hosts evolve), despite evidence that parasites are more likely to maximize $r$ in an epidemic (Bolker et al. 2010, Lion and Metz 2018); most studies assume infinite population size and do not consider stochasticity in the evolution of host or pathogen (but see Parsons et al. 2018), despite the large impacts that parasites can have on host population size (Frickel et al. 2016, Frickel et al. 2018).

\subsection*{Our contribution}

In most host-parasite systems that we have empirical data for we are either dealing with a novel introduction or a perturbed system that doesn't conform to the assumptions of AD or many of the simplifications that eco-evo models make (e.g. finite population size) in order to obtain analytic solutions. Rather than starting from a tradeoff and finding an ESS, here ask fairly general questions about what kind of community of parasites do we expect to see circulating, small pop size and drift, and not necessarily being optimal, what kind of community do we expect to see, mechanistic detail. This approach is similar to that of Loeuille and Loreau (2005) in their stochastic model of food webs. 

We build a simple host parasite coevolutionary model that... (in essence strives for a neutral (at least equivalent costs/benefits etc.)... we expect (my hypothesis) is that properties will emerge giving tolerance an advantage in a stochastic evolutionary model... Important to keep these expectations in mind in comparison in finite populations... Goal of making testable predictions (Loeuille and Loreau 2005).

We examine \mk{I am not so clear on this part, but I definitely think there is something here} the rate of fixation of a favorable adaptation (increase in resistance or tolerance) as a function of population size. Using an effect size that is likely to lead to fixation at \mk{some appropriate time frame}, we show how power to detect resistance is higher than for tolerance for effect sizes that have equivalent effects on host fitness. We show that range tolerance (variation within an individual over the course of infection) is especially difficult to detect and requires many observations within and among individuals (similar to Kain et al. 2015). Finally, we show that a positive correlation between resistance and tolerance will make it harder to detect differences in tolerance because of the reduced range over which parasite load is found. 

\section*{Methods}

