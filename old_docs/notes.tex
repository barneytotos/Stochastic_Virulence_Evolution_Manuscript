\section*{Intro stuff}

Cool statement about eco-evo feedbacks in resistance and tolerance evolution that would affect both the evolutionary paths: --- As part of this, do we have a sense about the \emph{ecological} aspects/feedbacks at reasonable effect sizes of resistance and tolerance? (e.g degree of resistance evolution will affect the growth rate of the focal species (e.g. by decreasing density due to costs of resistance --- Frickel et al. 2016), which can alter pathways for nutrients (Brunner et al. 2016), leading to cascades.

Despite work on host-parasite interactions for nearly a century \citep{Kostitzin1934, Ball1943}, the evolution of parasite exploitation and host defense remains a compelling puzzle. 

For example, the pathogen-intrinsic component is defined, among other things, by the expression of diverse virulence factors including toxins and transport proteins, as well as by tissue tropism and replication rates of the pathogen (Medzhitov et al. 2012). For example, in influenza a....

Can think of a parasites success as a multidimensional surface, where many genes interact to produce a parasites ability to transmit. Traditional tradeoff theory collapses these complex within-host dynamics into a black-box, where something is happening that results in a positive correlation between parasite virulence (the rate at which it kills its host) and transmission rate between hosts. Anderson and May (1982) showed that when this curve is decelerating (negative second derivative), there exists an intermediate level of virulence that maximizes parasite lifetime reproductive potential. Much work leaves it at this level, though many other works show how this relationship can arise from within-host dynamics. 

Yet, it is important to emphasize that virulence is a complex function that has at least two components. The pathogen-intrinsic component is defined, among other things, by the expression of toxins and other virulence factors, as well as by tissue tropism and replication rates of the pathogen (Medzhitov et al. 2012). For example, in influenza a....

In this chapter I examine the evolution of a suboptimally evolved parasite following its introduction into a new host population. I model parasites that evolve in aggressiveness, which is correlated with transmission following tradeoff theory of virulence evolution, but which must also evolve compensatory mutations to maintain transmission following changes in aggressiveness. I use a deterministic, reaction-diffusion framework as well as a stochastic, discrete time model in order to understand the effects that random mutations in finite populations has on a series of metrics associated with parasite evolution: time to a global adaptive peak, transient departures from optimal aggressiveness, and stochastic departures from the global optimum. I find that in both deterministic and stochastic models parasites evolve to the global fitness maximum. Smaller populations and higher mutation rates drive parasites to and increase transient departures from the optimal fitness trajectory up the peak. In both deterministic and stochastic models, the fastest ascent to the global optimum can lead to transient evolution of higher virulence.

Often, work on tradeoff theory considers the evolution of parasites in a homogeneous and unevolving static host population (for a review see Cressler et al. 2016), though progress has been made in recent years understanding both parasite evolution in heterogeneous host communities (Ganusov et al. 2002, Gandon 2004, Pugliese 2011, Osnas and Dobson 2012), and the co-evolution of host defenses and parasite exploitation (Roy and Kirchner 2000, Carval and Ferriere 2010, Best et al. 2014).

Orientation of the genetic distribution aligns with the surface (Lande and Arnold).

\section*{Results since making some progress on simulations}

First, I assume a pathogen evolves in the space of aggressiveness and efficiency, which translate into virulence and transmission rate. A host can evolve tolerance, which decreases the translation of aggressiveness into virulence and resistance, which decreases the effects of aggressiveness as well as transmission (by moving a parasite down its own tradeoff curve). Operating under these basic assumptions I explore the evolution of virulence and host defense with various assumptions about where biased mutations reside. The general idea here is to examine how a parasite evolves when it isn't already on its tradeoff frontier. What kind of basic assumptions change how a parasite evovles as it approaches the point where the major constraint to evolution is trading off alpha and beta? 

When there is a bias for increasing aggressiveness, a parasite can achieve higher R0, but will quickly drive itself to extinction, regardless of how mutation bias works in efficiency or how tolerance is evolving. Strong evolution of resistance can actually save the parasite from killing its host population in rare instances (evolution of resistance at the same speed as aggressiveness). Under these assumptions a parasite does not evolve like it is on a tradeoff curve because it can move in increasing alpha space and get increasing beta almost indefinitely if it is far from its frontier. Because of mutation selection balance the parasite cant ever actually reach its frontier, and will never evolve as if it is on its frontier. If somehow it manages to get there then it can (slowly) overcome the mutation bias problem and move towards the global optimum aggressiveness. These results are found in approximately the first 50 or so slides of the res tol explore keynote presentation.

When there is no bias in aggressiveness but bias in efficiency, a parasite operates as it would be expected to: moving towards the global optimum, determined by the point on the tradeoff curve of optimal aggressiveness that translates to the highest R0. A few things that arise here: the larger the biased mutation in efficiency (if those are drawn at the same time as a movement in aggressiveness) the longer it takes for a parasite to get in the vicinity of the global optimum aggressiveness. The larger the mutation bias the lower R0 it will obtain, though in the long long run it will eventually get to the global optimum at an infinite population, but not at a finite population. These results are found roughly between slides 50-80.

The interesting dynamics begin to arise when we start to consider a pathogen such as influenza, which tends to actually lose a bit of efficiency as it increases aggressiveness. In this case there are two opposing mechanisms of selection depending on the parameter space the parasite finds itself in, resulting in three distinct regions of parameter space. In the most interesting of these three spaces, the movement of the parasite will depend on the size of the mutation bias in efficiency. With no bias it will move in the direction of the global optimum. With even a small bias the parasite first evolves decreasing aggressiveness (because this actually results in an increase in efficiency), and once it reaches a ridge of equivalence in aggressiveness efficiency space (determined by the slope of the logit scale of these two parameters), the parasite will start to evolve up the tradeoff curve, resulting in an increase in aggressiveness (a good example of this is found on slide 108).

In general host resistance is just bad for the evolving parasite in the sense that it will take longer for it to keep the same R0, and can never get to a higher R0. On the other hand, there are many instances where host tolerance actually allows a parasite to get to its optimum faster (maintaining higher efficiency I THINK), and based on the slope of the gain in beta as a function of aggressiveness relative to alpha as a function of aggressiveness, a parasite will continue to evolve to higher aggressiveness (the former has to be bigger). However, once the slopes of these two curves are equal a parasite will stop evolving higher aggressiveness, will keep obtaining higher efficiency and thus higher R0, and a host will also continue to evolve tolerance resulting in higher fitness as well. All things being equal (which they are not) this means that tolerance evolution is better for both parties in the long run. 

A key question is HOW are all things not equal. Is the gradient towards resistance steeper in the early phases and do parasite and pathogen get sucked in the direction of evolving an arms race despite it not being the path to the global optimal fitness for either side? Not sure I can explore this now, but definitely later.

A key important question is what does all of this tell us? what are reasonable assumptions and what are not reasonable assumptions? Are there examples? How does time to peak, mutation bias, and population size interact to determine the long term stable equilibrium? 

An extremely key question is whether tolerance evolution, under some simple assumptions, actually changes where a parasite equilibrates in a finite population while it is evolving under even a small mutation bias --- This would be a key finding. Does evolution from above the optimum and below the optimum change how this key aspect changes?

Another adjustment that I recently made at the suggestion of JD is to reformulate in terms of parasite aggressiveness and parasite tuning, which together define parasite efficiency. Set up in this way, a parasite still moves along the path the an optimal tradeoff curve, but is also constrained by not moving too far in that direction because of mismatching in life history strategy. In this way the parasite moves in a zig-zag pattern towards its global optimum virulence.

