Svensson and Raberg 2010: For example, in the context of host–pathogen interactions, the pathogen could have suboptimal virulence if hosts evolve higher tolerance. This could lead to a form of coevolution where evolution of higher tolerance selects for more virulent pathogens. However, this form of coevolution would not be antagonistic and it should therefore result in a new stable equilibrium rather than cause open-ended dynamics.

Medzhitov et al. 2012: It is important to emphasize that virulence is a complex function that has at least two components. The pathogen-intrinsic component is defined, among other things, by the expression of toxins and other virulence factors, as well as by tissue tropism and replication rates of the pathogen. The host-intrinsic component is defined by its susceptibility (or tolerance) to the damage that can be caused by the pathogen or by the immune response it elicits. Both pathogenintrinsic and host-intrinsic characteristics can affect host fitness (48).

Eco-evo feedbacks including changes to density, which can decrease optimum virulence 

Can think 

Having weird problems with doc.tex so copy and pasting this hear in case things break

\section*{Introduction}

Host-parasite competition is a canonical example of a complex biological interaction: Parasites contain numerous genes that allow them to replicate, evade host immunity, compete with other organisms within the host, and ultimately transmit to new hosts; hosts strive to remove parasites and/or reduce the negative fitness impacts of parasite exploitation using a diversity of immune mechanisms \citep{Soaresetal.2017}. In the past four decades research has focused primarily on understanding parasite exploitation strategies by collapsing high-dimensional host-parasite interactions to two axes: parasite virulence (the rate of host mortality) and parasite transmission rate \citep{AndersonandMay1982, Ewald1983, AlizonandMichalakis2015, Cressleretal.2016}. When parasite transmission is a convex function of parasite virulence, parasites maximize fitness at an intermediate level of virulence \citep{AndersonandMay1982, Alizonetal.2009}. Intermediate virulence as a result of a virulence/transmission tradeoff has also been found in a small number of host-parasite systems including $\lambda$ phage in \emph{E. coli} \citep{Berngruberetal.2015}, HIV in humans \citep{Fraseretal.2007}, \emph{Ophryocystis} in monarch butterflies \citep{DeRoodeetal.2008}, Cauliflower Mosaic Virus in \emph{Brassica rapa} \citep{Doumayrouetal.2013}, and Myxoma virus in European rabbits \citep{AndersonandMay1982}.

This reductionist approach for modeling parasite evolution treats the complexities of parasite within-host dynamics as a ``black box'' (though see \citealt{AlizonandvanBaalen2005} who find that a convex tradeoff between parasite virulence and transmission arises from simple models of within-host dynamics). Though a strong simplification, these models have revealed a number of generalities about parasite evolution, applicable to a number of host-parasite systems (for a review see \citealt{Cressleretal.2016}). Yet, predictions for the evolution of specific parasites beyond general qualitative patterns can be difficult to make without illuminating at least part of the ``black box'' for a given host-parasite system \citep{AlizonandMichalakis2015}. 

More insight is to be gained from extensions to tradeoff theory that add complexity motivated by observations of specific host-parasite systems. Though this is not a new idea (see \citealt{Cressleretal.2016}), many potentially fruitful unexplored avenues of research building on the foundation of simple tradeoffs remain, including the stochastic evolution of parasites in small host communities following parasite invasion. Most existing modeling studies do not provide insight into the evolution of parasites and/or hosts during the course of an epidemic (but see \citealt{Bolkeretal.2010, LionandMetz2018, Parsonsetal.2018}), despite clear examples of rapid evolution following parasite invasion in many systems including WNV \cite{Beasleyetal.2003}, MYXV \citep{FennerandMarshall1957}, \emph{Plasmodium falciparum} \cite{Flemingetal.2018}, HIV \cite{Fraseretal.2007}, $\lambda$ phage \cite{Berngruberetal.2015}, \emph{phycodnaviridae} family dsDNA viruses (\citep{Frickeletal.2016, Frickeletal.2018}, and \emph{Pasteuria ramosa} \citep{Duffyetal.2009}.

Most modeling work on parasite evolution uses a separation of ecological (epidemiological) and evolutionary time scales (e.g. using the \emph{Adaptive Dynamics} framework which also assumes small effect mutations \citealt{Diekmann2002}), which is a departure from the reality of rapid evolution during an epidemic or as a result of environmental change such as seasonality \citep{Lion2018} (however, see \citealt{Bolkeretal.2010}, \citealt{LionandMetz2018}). Most studies also assume infinite population size and do not consider stochastic evolution pathogen, despite the large impact that parasites can have on host population size \citep{Frickeletal.2016, Frickeletal.2018}. While \citet{Parsonsetal.2018}, who derive analytic solutions to parasite evolution in finite populations using \emph{Stochastic Adaptive Dynamics}, does more closely resembles parasite evolution in practice, they assume rare and small-effect mutations. 

If a finite-population stochastic model that allows for ecological and evolutionary feedbacks is the closest to biological reality, why do we not turn to these types of models sooner? One reason is that researchers often are in search of analytical solutions; there is currently a lack of mathematical theory to support analytic solutions for the evolution of finite populations on ecological time scales. \citet{Lion2018} writes: ``Analytical progress is possible only through additional methodological assumptions, generally taking the form of a separation of time scales''. Similarly, analytic solutions are more easily generalizable; solutions from stochastic simulations are likely to be tied more closely to specific systems. It is unclear, however, how much of a drawback this is, as many researchers are calling for more of a case-study approach in the field \citep{BullandLauring2014, AlizonandMichalakis2015, Cressleretal.2016}. Finally, stochastic models can be difficult to parameterize, and qualitative dynamics (e.g. bifurcations) are harder to pin down accurately. Despite these difficulties, stochastic simulations are likely to be the most appropriate model for capturing biologically reality. Progress can still be made using simulations, especially when a deterministic model can be used as a null-model against which to compare results, such as expected magnitude of transient departures from the most direct evolutionary path to the global optimum. 

Here we present a model for the evolution of parasite exploitation strategy that introduces a second trait axis on which parasites evolve: a quantitative trait representation of investment in compensatory traits (e.g. secondary virulence factors) that affect the ability for a parasite to effectively translate ``virulence'' into transmission. In classic tradeoff theory, parasite virulence is either discussed as a trait onto itself, or is described as a result of parasite replication rate \citep[e.g.][]{AlizonandvanBaalen2005}. It is assumed that a change in parasite virulence results in a correlated change in transmission following an assumed tradeoff curve function (often either a power-law or sigmoidal function \citealt{AlizonandvanBaalen2005, Bolkeretal.2010, Kainetal.2018}. However, an increase in parasite replication rate may only lead to an increase in transmission rate following changes in other genes. For example, changes to the MYXV virulence factor M148R, which helps to downregulate the host immune system, are needed for an increase in MYXV replication rate to increase transmission, because a secondary side effect of an increased replication rate is the faster activation of the host immune system \citealt{Blanieetal.2009}. Even in the absence of specific examples, it is not difficult to imagine that an increase in investment in replication rate, for example, leads to a decrease in per capita infection efficiency. To recover this per capita efficiency and further increase \emph{total} efficiency, adjustments in secondary traits are needed.

While it is conceivably possible to model the evolutionary dynamics of multiple individual virulence factors, to our knowledge no system is simple enough nor is enough currently known about the function of all protein products of a specific parasite for this to be a fruitful approach. Instead, we capture the general idea of the need for ``compensatory mutations'' by adding a single additional trait axis to the classic two-axis tradeoff theory paradigm. We model the need for compensatory mutations using a quantitative trait we call parasite parasite \emph{tuning}, which evolves to allow a parasite to maximize its transmission potential defined by the tradeoff curve. We define parasite \emph{frontier transmission} as the transmission defined by the tradeoff curve at a given level of parasite virulence, where parasites evolve on the axis of replication rate; parasite \emph{realized transmission} is less than or equal to frontier transmission. When a parasites' investment in tuning perfectly matches its investment in aggressiveness realized transmission equals frontier transmission, otherwise realized transmission is lower than frontier transmission.  

We examine the evolution of a parasite that invades a new host population that is initially beneath its tradeoff frontier. \citet{Alizonetal.2009} mention in passing that parasites beneath their frontier will first evolve to their frontier, and then to the tradeoff optima; however, no additional discussion is given here, or to our knowledge elsewhere, on the evolutionary trajectory a parasite takes to reach its global optimum. However, as many have pointed out, tradeoff theory is a simple model \citep{BullandLauring2014, AlizonandMichalakis2015, Cressleretal.2016}; it is not only unlikely that a parasite will evolve only along the tradeoff curve, it is unlikely that it will always find itself on the tradeoff curve, or for that matter, even be able to make it to its tradeoff curve. The tradeoff curve is by definition an evolutionary frontier; it is a certainty, especially for a parasite that has experienced a host jump, that a parasite may be far from its frontier.

Using this model for the mechanics of parasite exploitation, we examine the ecological mechanics of infection and evolutionary mechanics of parasite exploitation strategy using three different approaches. Fundamentally, we are interested in a biologically realistic scenario of parasite evolution in finite populations; however, these simulation-based models can be intractable at times. We use two different deterministic modeling frameworks to which we compare the results of our stochastic model: a deterministic reaction-diffusion (differential equation based) model (RD) that captures both selection (advection) and mutation (diffusion) and an adaptive dynamics representation of the problem (AD). Using these three models we aim to understand the effects of small populations and stochasticity on transient evolution, time to the adaptive peak, and periodic departures of the population from its adaptive peak. We examine when and how interesting biological phenomena arise from different modeling assumptions; when is it worth it to worry about the additional layers of complexity and headache that a discrete time stochastic model affords you over a simpler deterministic model, or an adaptive dynamics model? Are we likely to get something wrong when we use one framework over another?

\section*{Methods}

We examine a parasite that is subject to a tradeoff between virulence and transmission \citep{Andersonetal.1982, Ewald1983}, where transmission rate ($\beta$) is assumed to be an increasing function of host mortality rate (virulence: $\alpha$). When $\beta$ is a convex function of $\alpha$ ($\frac{\partial^2 \beta}{\partial \alpha^2} < 0$), a single intermediate value of $\alpha$ maximizes $\rzero$ \citep{Alizonetal.2009}. Here we use a power-law function, a commonly used decelerating function to model the relationship between $\alpha$ and $\beta$ \citep{AlizonandvanBaalen2005, Bolkeretal.2010}: 

\begin{equation*}
\beta(\alpha) = c\alpha^{\frac{1}{\gamma}},
\end{equation*}

\noindent where $\gamma$ controls the shape of the curve and $c$ is a scaling factor. For a power-law function $\rzero$ is given by:

\begin{equation*}
\rzero = \frac{c\alpha^{\frac{1}{\gamma}}}{\mu + \alpha}
\end{equation*}

\noindent where $\mu$ is the background rate of host mortality. The $\rzero$ of a parasite constrained by only a power-law tradeoff is obtained at $\alphastar = \frac{\mu}{(\gamma - 1)}$. In our model, however, parasite transmission is also affected by a match between what we call parasites aggressiveness (which underlies $\alpha$ and can be thought of as the log of replication rate---we return to this below), and a variety of what refer to as ``compensatory'' virulence factors. For example, a modification of virulence factors that downregulate a hosts' immune system which gets upregulated as parasite replication rate increases as seen in MYXV (M148R factor experiences mutations to increase efficiency as MYXV replication rate increases: \citealt{Blanieetal.2009}. We imagine here that sufficient investment in compensatory proteins are what allows a parasite to realize maximum transmission at a given level of aggressiveness. We model a parasites' total investment in these secondary response traits using a single quantitative trait axis which we term a parasites ``tuning''. 

We define $\beta$ given by the power-law tradeoff as frontier ($\betaf$): a transmission maximum that is achievable only when parasite tuning is perfectly matched to parasite aggressiveness. Parasite realized beta ($\betar$) is the transmission a parasite experiences given its trait value for both aggressiveness and tuning. We calculate $\betar$ as $\betaf$ multiplied by parasite efficiency ($\omega$), which is given by:

\begin{equation*}
\omega = e^{\frac{-(\phi - \theta)^2}{\delta}},
\end{equation*}

where parasite aggressiveness is given by $\phi$, parasite tuning is given by $\theta$, and $\delta$ is a scaling factor that controls the severity of the cost of a mismatch between $\phi$ and $\theta$. Thus, $\betar = \betaf * \omega$ only when $\betar = \betaf$, which occurs only if $\phi = \theta$ (i.e. a perfectly tuned parasite). In this model parasite $\rzero$ is given by:

\begin{equation*}
\omega \frac{\beta(\alpha)}{\alpha + \gamma + \mu}.
\end{equation*}

While parasite fitness is defined by $\rzero$, which is dependent on $\alpha$ and $\beta$, aggressiveness ($\phi$) and tuning ($\theta$) are the traits that evolve. We assume that parasite aggressiveness is the inverse-logit of $\phi$ ($\alpha = \frac{e^{\phi}}{e^{\phi} + 1}$). We use a logistic scale for two reasons: First, the non-linear logit scale is a convenient scale on which to model mutational that have little effect when a parasite has either a low or high trait value; second, the logistic scale maps an unbounded mutational scale to (0, 1), which allows us to map parasite aggressiveness and tuning to probabilities of transmission and host mortality for our discrete time stochastic model.

In reality, parasite virulence and transmission is a joint function of parasite exploitation strategy and host defense strategy; following parasite invasion, parasite exploitation and parasite defense are known to co-evolve \citep{FennerandMarshall1957, CarvalandFerriere2010, Bestetal.2014}. Here we assume parasite invasion of naive host populations with no prior evolutionary history with the parasite. While we do no explicitly model host immune systems, a static level of total host defensive capabilities is intrinsically captured in our model using parasite tuning: a change in parasite aggressiveness can be thought to decrease parasite fitness because of a mismatch to a given host system, at which parasites must ``retune'' to be able to return to a maximal level of transmission. 

Using this mechanistic description of host exploitation and fitness, we examine parasite evolution using three different methods for modeling parasite transmission: a discrete time, stochastic, finite population model (DTS), a reaction-diffusion, differential equation model (RD) and an adaptive dynamics (AD) model. Our primary interest is on stochastic evolutionary dynamics in small populations; however, results from stochastic simulations can be difficult to interpret on their own. We use results from the RD model, which allows for eco-evolutionary feedbacks, but is deterministic and assumes an infinite population size, and the AD model which assumes a separation ecological and evolutionary time scales, and small-effect mutations, to examine the effects of small populations and stochasticity on transient evolution, time to the adaptive peak, and periodic departures of the population from its adaptive peak. We begin with a description of the DTS model and then describe what simplifications are required to fit the assumptions of RD and AD.

\subsection*{Discrete time, stochastic, finite population model (DTS)}

We model parasite transmission using an SISD model, defined by three classes of host: Susceptible (S), infected (I), and dead (D). Infected hosts either recover from infection, at which point they become susceptible once again, or die and become removed entirely from the population. We assume a homogeneous host population. Parasite strains vary in both aggressiveness and tuning and thus virulence and transmission.

In each time step the following occur: S and I hosts die with probability given by the host background death rate \emph{d}; I hosts die given the virulence ($\alpha$) of the strain they are infected with; S hosts reproduce in a density dependent manner, where the probability of reproduction is inversely proportional to the ratio of the sum of living S and I hosts to the starting population size (I hosts are assumed to be unable to reproduce); I hosts recover with probability $\gamma$; a proportion of S hosts \emph{escape} infection with probability equal to:

\begin{equation*}
\prod_{i} (1 - \beta_{i}),
\end{equation*}

\noindent while the susceptible hosts that do become infected are infected with strain \emph{i} with probability \emph{approximately} equal to:

\begin{equation*}
p_{i} = \frac{n_{i}R_{0i}}{\sum_{j}n_{j}R_{0j}}.
\end{equation*}

\noindent We assume a mutation occurs in parasite strain \emph{i} during transmission with probability $\mu$, and that mutations have an additive effect on the logistic scale. Mutational effect sizes for both traits are drawn from a multivariate Normal distribution with a mean of zero, positive variance, and zero covariance. That is:

\begin{equation*}
\left.\begin{matrix}
 & \phi_{i} \rightarrow \phi_{i} + MVN(0, \Sigma_{1,1})  \\ 
 & \theta_{i} \rightarrow \theta_{i} + MVN(0, \Sigma_{2,2}) 
\end{matrix}\right\} | B(1, \mu) = 1
\end{equation*}

Newly birthed and recovered hosts are not available to be infected until the next time step and newly infected hosts cannot die, recover, or infect S hosts until the next time step. This process proceeds for a specified length of time or until all hosts or the parasite population become extinct. 

\subsection*{Reaction diffusion model}

The model described above can be written using a differential equation framework, using rates instead of probabilities:

%\begin{equation*}
\begin{align*}
& \frac{\mathrm{d} S}{\mathrm{d} t} = bS - \mu S - S \int_{0}^{\infty}  \int_{0}^{\infty} \beta(\alpha(\phi, \theta, t)))i(\phi, \theta, t)d\phi d\theta + \gamma \int_{0}^{\infty}  \int_{0}^{\infty} i(\phi, \theta, t)d\phi d\theta \\ 
& \frac{\partial i}{\partial t} = [S\beta(\alpha(\phi, \theta, t)) - (\alpha + \mu + \gamma)]i(\phi, \theta, t) + D\frac{\partial^2 i}{\partial \phi^2} + D\frac{\partial^2 i}{\partial \theta^2},
\end{align*}
%\end{equation*}

\noindent where the second derivative (diffusion) terms capture mutation in the two quantitative traits of interest. To remove the integrals in these equations and for efficient solving of this system, we discretized aggressiveness and tuning into an m x m matrix, creating a box-car PDE model \citep{DeRoos1988}. Each cell of this matrix represents a combination of trait values, and the number in each cell representing the number (proportion) of individuals infected with a strain with that combination of trait values. Thus, diffusion on this matrix represents mutation from one trait value to another. 

We solved this system of differential equations was solved using the {\tt ode} function in the {\tt deSolve} package in {\tt R}; diffusion was implemented using the {\tt tran.2d} function in the {\tt ReacTran} package using a zero-flux boundary condition (no loss of infected individuals due to mutation). Hosts were assumed to have the same recovery rate from infection for all parasite strains (for the evolution of recovery see \citealt{AndersonandMay1982, AlizonandvanBaalen2005}). 

\subsection*{Adaptive dynamics model}

When ecological dynamics are assumed to be fast relative to evolutionary dynamics (a new parasite mutant only invades at ecological equilibrium), and mutational effect size is small, it has been shown that parasites evolve to optimize $\rzero$ \cite{Dieckmann2002, KeelingandRohani2008, Lion2018} in simple epidemiological models (e.g. SIS, SIR). By definition, once at ecological equilibrium a resident strain has an $\rzero = 1$. A new mutant is able to invade if it has an $\Reff > 1$ (a single host infected with a mutant strain infects greater than one new host at the equilibrium number of S set by the resident strain). In simple epidemiological models \cite{Dieckmann2002, KeelingandRohani2008, Lion2018}, an $\Reff > 1$ corresponds to an $\rzero > 1$. Starting with an arbitrary stain, mutant strains with $\Reff = 1 + \epsilon$ invade and fixate until no new mutant strains have $\Reff > 1$. This process can be used to find an optimal parasite strategy, though it is is not guaranteed that the optimum found is a global optimum \citep{Dieckmann2002}.

Our model produces a fitness landscape with a single global maximum and no local maxima for all parameter values. AD can therefore be used to show how evolution proceeds in our system under the assumptions of a separation of time scales and small mutation size. We implement a crude form of AD by relying on the criteria that a mutant strain with $\rzero$ greater than the resident strain will displace the resident strain. Using mutations with an additive effect of $\pm 0.005$ on the logistic scale, we allow mutant strains to replace the resident strain until no mutant has an $\rzero$ > than the resident strain's $\rzero$. We consider parasite evolution when a parasite can have a simultaneous mutation in both aggressiveness and tuning (which we assume in both the DTS and RD models), or a mutation in only a single trait at a time. Because it is possible that a mutational change of 0.005 in either aggressiveness or tuning can lead to a higher $\rzero$, we assume that the parasite with the largest $\rzero$ invades.

\subsection*{Model outcomes}

Our primary aim is to understand the dynamics of the SPS model; we use the results from the RD and AD models to help us clarify the effects of small populations and stochasticity. We focus on three outcomes: transient patterns in parasite virulence, the time it takes a parasite to reach its global optimum in units of parasite and host generations, and the size of the long-term virulence distribution (due to mutation/selection balance and/or stochastic departures from the global optimum). We ran all three models across a range of parameters; we present all simulation parameter values in Table~1. 

\begin{table}[H]
\centering
\caption{Parameter values used in the DTS (discrete time, stochastic, finite population) RD (reaction-diffusion), and AD (adaptive dynamics) models. Parameter values that are not used in a given model are written as ``--''. Starting values for $\phi$ and $\theta$ for the RD model are closer to 0 on the logit scale because of numeric instability at more extreme starting conditions. Values for the power-law scaling parameter (\emph{c}) was increased in the RD model to insure an $\rzero > 1$ at extreme combinations of $\phi$ and $\theta$ with a $\delta$ of 10. An increase in \emph{c} increases $\rzero$ but does not affect the shape of the fitness surface. } \label{tab:model_details} 
\resizebox{\columnwidth}{!}{
\bgroup
\begin{tabular}[c]{|R{2.50cm}:R{4.00cm}:R{4.00cm}:R{4.20cm}|}
\hline
     \textbf{Parameter}        &
     \textbf{DTS model}        &
     \textbf{RD model}         & 
     \textbf{AD model} 	
\\
\hline
\rowcolor{vlg}
$\delta$     & 
10, 30, 50   &
10, 30, 50   &
10, 30, 50   
       
\\
$\mu$                                            & 
0.001, 0.005, 0.025                              &
--                                               &
1 or 2 mutations in each new mutant, 
occurs after ecological equilibrium is reached   
       
\\
\rowcolor{vlg}
Mutational $\Sigma$                        & 
0.05, 0.01, 0.25                           &
0.025 * 2.5\textsuperscript{0, 1, 2, 3}    &
Constant mutational size of 0.005          
       
\\
Population Size    & 
200, 600, 1800     &
--                 &
--                 
       
\\
\rowcolor{vlg}
Starting [$\phi$, $\theta$]                                                         & 
[logit(0.03), logit(0.97)], [logit(0.97), logit(0.03)], [logit(0.03), logit(0.03)]  & 
[logit(0.80), logit(0.20)], [logit(0.20), logit(0.80)], [logit(0.20), logit(0.20)]  &
[logit(0.03), logit(0.97)], [logit(0.97), logit(0.03)], [logit(0.03), logit(0.03)]  
       
\\
Power-law \emph{c} & 
0.75     &
2-20     &
0.75     
       
\\
     \rowcolor{vlg}
Power-law \emph{$\rho$} & 
2     &
2     &
2     
       
\\

\hline

\end{tabular}
\egroup
}
\end{table}

\clearpage


\section*{Results}

We present results for all models for a single parameter set visually using the movement of the parasite population on the fitness surface from: 250 stochastic runs as well as the RD and AD solutions. We show the effects of individual parameters on our metrics of interest, for example, the effects of $\delta$ on transient virulence, by plotting each metric against a range of values for a single parameter at the median of all other parameters. We save an analysis of the interactions of parameters for a later date.

\subsection*{Climbing the adaptive landscape}

\subsection*{Parameter dependence}

I find that in both deterministic and stochastic models parasites evolve to the global fitness maximum. Smaller populations and higher mutation rates drive parasites to and increase transient departures from the optimal fitness trajectory up the peak. In both deterministic and stochastic models, the fastest ascent to the global optimum can lead to transient evolution of higher virulence.

\section*{Discussion}

Previous work has shown that parasites can parasites can evolve transient levels of suboptimal virulence. Here I find that parasites may experience either transient periods of higher than optimum or lower than optimum virulence depending on starting conditions and the shape of the tradeoff curve and severity of mismatch of aggressiveness and tuning.

\subsection*{Caveats}

Lots of numeric instabilities with the deterministic model based on size of R0, steepness of gradient, and level of diffusion. Distribution can \emph{jump} over the steep part of the gradient and end up in a place that seems to be a numeric issue and not the true path that the distribution will take up the surface.

Need to think very deeply about scales. Mutation on one scale, but selection on another can produce some pretty odd dynamics. Note of Bens about mutation on the log scale and selection on the linear scale, which leads to exponentially increasing fitness.

\subsubsection*{Host defense evolution}

Competition between a host and parasite often generates strong selection pressure on parasite exploitation strategy and host defense strategy. Parasites are assumed to be able to adjust their aggressiveness, while hosts can invest in resistance (reducing parasite load) or tolerance (decreasing fitness costs for a given pathogen load) (Boots 2008, Adelman and Hawley 2017). 

While I focus on parasite evolution here, I briefly consider the evolution of parasite strategy in populations with evolved resistance (host strategy to reduce pathogen aggressiveness and thus transmission) and/or tolerance (host strategy to reduce the fitness cost of infection) (for reviews on resistance and tolerance see \citealt{SchneiderandAyres2008, AyresandSchneider2012, KutzerandArmitage2016}). Moving forward, we are going to extend this model to examine the co-evolution of parasite and host strategies; we use parasite evolution against a range of static host strategies as a springboard for future work. Predicting evolved levels of parasite aggression and host investment in resistance and tolerance is important for determining host mortality and recovery rate as well as parasite transmission rate, and can help to determine responses to ... Resistance evolution in hosts is predicted to generate Red Queen cycles... while tolerance evolution is expected to lead to selective sweeps but does not impact... Because of these different feedbacks on both parasite load and prevalence of host infection, the type of defense that a host invests in can alter the evolutionary trajectory of the host-parasite interaction... 

It will be very important to extend this framework to consider the co-evolution of hosts and parasites in discrete time stochastic models.

The evolution of host defense responses to parasite exploitation have been studied using mathematical models for decades (Anderson and May 1982, Roy and Kirchner 2000, Best et al. 2014). This vast modeling literature has historically considered host resistance (reducing pathogen load) as the primary mechanism of host defense; more recent theoretical and empirical studies (theoretical: Carval and Ferriere 2010, Best et al. 2014; empirical: Raberg et al. 2007, Athanasiadou et al. 2015, Rangan et al. 2016) have considered host tolerance (decreasing the impact of a given pathogen load on host fitness) as a host strategy for mitigating the effects of pathogen infection. (The plant-pathogen literature has focused on tolerance for nearly a century [Foroni et al. 2013], but the animal-pathogen literature has only recently turned in this direction.) Most models consider host defense evolution in response to a non-evolving parasite, or the coevolution of a single host defense (resistance or tolerance) and pathogen virulence (the loss of fitness, or rate of mortality, of infected hosts); however, at least one study has employed a full coevolutionary model, considering evolution of both resistance and tolerance of hosts as well as evolution of parasite virulence (Carval and Ferriere 2010). 

In addition to variation among studies in what traits are assumed to be evolving, models vary in their assumptions about the genetic architecture of host-parasite interactions; the time scale on which ecology (epidemiology) and evolution occur; whether or not the effects of finite population size are considered; and the mode of selection. For a good description of the full range of genetic assumptions (gene-for-gene: GFG; matching allele: MA; and quantitative genetic: QG approaches) see Best et al. (2014). For a review on options for modeling evolution see Lion 2018; and for a more focused review on optimization criteria see Lion and Metz 2018.